\input{macro.tex}

\begin{document}

\noindent La passe d'alpha-renommage utilise la technique suivante : On constitue un environnement de renommage, c'est � dire une pile de couple de variables, la premi�re �tant la variable � renommer, la seconde la valeur de cette variable apr�s renommage. Lorsque qu'une nouvelle variable $x$ est d�fini dans le code source, on ajoute cette variable � la pile, en l'associant � une nouvelle variable $y$.\\
\\
Cette nouvelle variable $y$ est d�finie selon le nom de la variable qu'elle remplace, et de sa place dans la pile : En effet, si $x$ est d�j� pr�sente dans la pile et est associ�e � $y'$, on veux que $y \neq y'$. Pour ce faire, on inclut dans le nom de la variable l'information de son num�ro dans la pile. Cela nous assure que pour deux couples distincts $(x,y)$ et $(x,y')$ dans la pile, $y \neq y'$.\\
\\
Mais il faut �galement faire attention � ce que si $x \neq x'$ et que les couples $(x,y)$ et $(x',y')$ sont dans la pile, alors on a toujours $y \neq y'$ (c'est la moindre des choses pour �viter les captures de variables).\\
\\
Voici la m�thode que l'on utilise pour le renommage : 

\begin{itemize}
	\item Si la variable $x$ est d�finie dans le code source, et que aucun couple $(x,y)$ n'est d�j� pr�sent dans la pile, alors on ajoute $(x,x \cdot x \cdot 0)$ au sommet de la pile, o� $\cdot$ repr�sente la concat�nation des noms de variables.
	\item Si au contraire, un couple $(x,y)$ est d�j� pr�sent, et que $y$ "porte" le num�ro $n$\footnote{$(x,x)$ porte le num�ro 0...}, on ajoutera $(x, x \cdot x \cdot\ent{n+1})$ au sommet de la pile, o� $\ent{n}$ est l'�criture en base 10 de $n$.\\
\end{itemize}

\noindent Il est trivial de constater que si $(x, y)$ et $(x, y')$ sont deux couples distincts de la pile, alors $y \neq y'$.\\
D'autre part, si $(x,y)$ et $(x', y')$ sont deux couples distincts de la pile avec $x \neq x'$, alors montrons que $ y \neq y'$. On raisonne sur la taille $\abs{ \bullet }$ de $y$ et $y'$ et on suppose que $x \cdot x \cdot \ent{n} = x' \cdot x' \cdot \ent{n'}$. Supposons, par sym�trie, que $\abs{x} \leqslant \abs{x'}$. Donc on peut dire que $x' = x \cdot u$. L'�galit� devient alors

$$ x \cdot \ent{n} = u \cdot x \cdot u \cdot \ent{n'} $$

\noindent Soit $a$ la premi�re lettre de $x = a \cdot z$. On a donc l'�galit� 

$$ a \cdot z \cdot \ent{n} = u \cdot a \cdot z \cdot u \cdot \ent{n'}.$$

\noindent Si $u = \varepsilon$, on a fini car alors $x = x'$. Sinon, $u$ commence par $a$. Or, selon la derni�re �quation, on sait que $u$ est un sous-mots de $\ent{n} \in \{0, ..., 9\}^*$. Donc $a \in \{0, ..., 9\}$, ce qui est impossible car $x$ est un nom de variable et ne peut par cons�quent pas commenc� par un chiffre. CQFD.








\end{document}
