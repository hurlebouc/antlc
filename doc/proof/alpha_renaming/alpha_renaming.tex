%!TEX TS-program = latex

%\documentclass[compress]{beamer}
%\useoutertheme[subsection=false]{miniframes}
%\usepackage{beamerthemesplit}
%\useoutertheme[]{split}
%\usetheme{Copenhagen}

\documentclass{article}

\usepackage[francais]{babel}
\usepackage{amsmath}
\usepackage{stmaryrd}
\usepackage{listings}
\usepackage{moreverb}
\usepackage{amsfonts,amssymb,verbatim}
\usepackage{amsthm}
\usepackage{mathrsfs}
\usepackage{graphicx}
\usepackage{pstricks,pst-node}
\usepackage[latin1]{inputenc}
\usepackage{color}
\usepackage[T1]{fontenc}
\usepackage{lastpage}
\usepackage{calc}

\lstset{ basicstyle={\ttfamily \small}, language=c, keywordstyle=\color{blue}, frame=lines, backgroundcolor=\color{gris}, breaklines = true, showstringspaces=false}


%%%%
\def\sqb{\hbox{\hskip5pt\vrule width4pt height6pt depth1.5pt%
\hskip1pt}}
\def\qed{\ifmmode\hbox{\hfill\sqb}\else{\ifhmode\unskip\fi%
\nobreak\hfil
\penalty50\hskip1em\null\nobreak\hfil\sqb
\parfillskip=0pt\finalhyphendemerits=0\endgraf}\fi}
\def\sqw{\hbox{\rlap{\leavevmode\raise.3ex\hbox{$\sqcap$}}$%
\sqcup$}}
\def\cqfd{\ifmmode\sqw\else{\ifhmode\unskip\fi\nobreak\hfil
\penalty50\hskip1em\null\nobreak\hfil\sqw
\parfillskip=0pt\finalhyphendemerits=0\endgraf}\fi}
%%%%


\newcommand{\imp}[1]{\textit{\textbf{#1}}}

 
\newcommand{\R}{\mathbb{R}}
\newcommand{\C}{\texttt{C}}
\newcommand{\Q}{\mathbb{Q}}
\newcommand{\Z}{\mathbb{Z}}
\newcommand{\N}{\mathbb{N}}
\newcommand{\K}{\mathbb{K}}
\renewcommand{\P}{\mathfrak{P}}
\newcommand{\pere}{\mathtt{pere}}
\newcommand{\pool}{\mathtt{pool}}
\newcommand{\num}{\mathtt{num}}
\newcommand{\pop}{\mathtt{pop}}
\newcommand{\push}{\mathtt{push}}
\newcommand{\T}{\mathfrak{T}}
\newcommand{\n}{\mathbf{n}}



\newcommand{\ssi}{si et seulement si }
\newcommand{\re}{\mbox{Re} }
\newcommand{\im}{\mbox{Im} }
\newcommand{\vect}{\mbox{vect} }
\newcommand{\rg}{\mbox{rg} }
\newcommand{\Ker}{\mbox{Ker} }
\newcommand{\tr}{\mbox{tr} }
\newcommand{\Sp}{\mbox{S}_{p} }
\newcommand{\Quad}{\mbox{Quad} }
\newcommand{\grad}{\overrightarrow{\mbox{grad}} }
\newcommand{\rot}{\overrightarrow{\mbox{rot}} }

\newcommand{\bea}{\texttt{Beaengine} }
\newcommand{\dis}{\texttt{DISASM} }
\newcommand{\out}{\textit{out} }
\newcommand{\fdis}{\texttt{int:Disasm(*DISASM)} }
\newcommand{\jb}{\textit{junk byte} }
\newcommand{\jbs}{\textit{junk bytes} }
\newcommand{\noeud}{n\oe ud }
\newcommand{\noeuds}{n\oe uds }
\newcommand{\algo}{algorithme }
\newcommand{\algos}{algorithmes }
\newcommand{\bs}{\textit{base set} }
\newcommand{\pb}{\textit{pullback }}
\newcommand{\id}{\mathsf{id}}
\newcommand{\Id}{\mathsf{Id}}

\newcommand{\Set}{\mathsf{Set}}
\newcommand{\define}{\overset{\mathrm{def}}=}

\newcommand{\diapo}[2]{\frame{\frametitle{#1} #2 }}

\definecolor{gris}{rgb}{0.9,0.9,0.9} 
\definecolor{bleuclair}{rgb}{0.7,0.7,1}


\newcommand{\abs}[1]{\left| #1 \right|}
\newcommand{\norme}[1]{\left\Vert #1 \right\Vert}
%\newcommand{\tribarre}[1]{\left\Vert \! \! \: \left| #1 \right| \! \!  \: \right\Vert}
%\newcommand{\dint}[1]{\displaystyle{\int} \! \!  \! \! \displaystyle{\int_{ #1}} }
%\newcommand{\tint}[1]{\displaystyle{\int} \! \!  \! \! \displaystyle{\int} \! \!  \! \! \displaystyle{\int_{#1}} }
\newcommand{\tribarre}[1]{\left\VERT #1 \right\VERT} % avec le package fourier
\newcommand{\dint}[1]{\displaystyle{\iint_{#1}}} % avec le package fourier
\newcommand{\tint}[1]{\displaystyle{\iiint_{#1}}} % avec le package fourier
\newcommand{\scal}[1]{\left\langle #1 \right\rangle}
\newcommand{\ent}[1]{\left\llbracket#1\right\rrbracket}
\newcommand{\somme}[2]{\displaystyle{\sum_{#1}^{#2}}}
\newcommand{\intergale}[2]{\displaystyle{\int_{#1}^{#2}}}
\def\jfrac#1#2{\raisebox{2pt}{$#1$}/\raisebox{-2pt}{$#2$}}   %%%%  jolie fraction
\newcommand{\barre}[1]{\overline{#1}}

\def\dedge{\ncline[linestyle=dashed]}
\def\couleur{\ncline[linecolor=red]}

\psset{arrows=->,shortput=nab}

\newcommand{\repr}[1]{\ent{#1}}


\begin{document}

\noindent La passe d'alpha-renommage utilise la technique suivante : On constitue un environnement de renommage, c'est � dire une pile de couple de variables, la premi�re �tant la variable � renommer, la seconde la valeur de cette variable apr�s renommage. Lorsque qu'une nouvelle variable $x$ est d�fini dans le code source, on ajoute cette variable � la pile, en l'associant � une nouvelle variable $y$.\\
\\
Cette nouvelle variable $y$ est d�finie selon le nom de la variable qu'elle remplace, et de sa place dans la pile : En effet, si $x$ est d�j� pr�sente dans la pile et est associ�e � $y'$, on veux que $y \neq y'$. Pour ce faire, on inclut dans le nom de la variable l'information de son num�ro dans la pile. Cela nous assure que pour deux couples distincts $(x,y)$ et $(x,y')$ dans la pile, $y \neq y'$.\\
\\
Mais il faut �galement faire attention � ce que si $x \neq x'$ et que les couples $(x,y)$ et $(x',y')$ sont dans la pile, alors on a toujours $y \neq y'$ (c'est la moindre des choses pour �viter les captures de variables).\\
\\
Voici la m�thode que l'on utilise pour le renommage : 

\begin{itemize}
	\item Si la variable $x$ est d�finie dans le code source, et que aucun couple $(x,y)$ n'est d�j� pr�sent dans la pile, alors on ajoute $(x,x \cdot x \cdot 0)$ au sommet de la pile, o� $\cdot$ repr�sente la concat�nation des noms de variables.
	\item Si au contraire, un couple $(x,y)$ est d�j� pr�sent, et que $y$ "porte" le num�ro $n$\footnote{$(x,x)$ porte le num�ro 0...}, on ajoutera $(x, x \cdot x \cdot\ent{n+1})$ au sommet de la pile, o� $\ent{n}$ est l'�criture en base 10 de $n$.\\
\end{itemize}

\noindent Il est trivial de constater que si $(x, y)$ et $(x, y')$ sont deux couples distincts de la pile, alors $y \neq y'$.\\
D'autre part, si $(x,y)$ et $(x', y')$ sont deux couples distincts de la pile avec $x \neq x'$, alors montrons que $ y \neq y'$. On raisonne sur la taille $\abs{ \bullet }$ de $y$ et $y'$ et on suppose que $x \cdot x \cdot \ent{n} = x' \cdot x' \cdot \ent{n'}$. Supposons, par sym�trie, que $\abs{x} \leqslant \abs{x'}$. Donc on peut dire que $x' = x \cdot u$. L'�galit� devient alors

$$ x \cdot \ent{n} = u \cdot x \cdot u \cdot \ent{n'} $$

\noindent Soit $a$ la premi�re lettre de $x = a \cdot z$. On a donc l'�galit� 

$$ a \cdot z \cdot \ent{n} = u \cdot a \cdot z \cdot u \cdot \ent{n'}.$$

\noindent Si $u = \varepsilon$, on a fini car alors $x = x'$. Sinon, $u$ commence par $a$. Or, selon la derni�re �quation, on sait que $u$ est un sous-mots de $\ent{n} \in \{0, ..., 9\}^*$. Donc $a \in \{0, ..., 9\}$, ce qui est impossible car $x$ est un nom de variable et ne peut par cons�quent pas commenc� par un chiffre. CQFD.








\end{document}
